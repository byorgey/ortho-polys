% -*- compile-command: "pdflatex bracelet-cycle-index.tex" -*-
\documentclass{article}

\usepackage{amsmath}
\usepackage{xspace}

\newcommand{\ie}{\emph{i.e.}\ }

\renewcommand{\S}{\mathcal{S}}

\DeclareMathOperator{\Fix}{Fix}

\begin{document}

\section{Preliminaries}
\label{sec:preliminaries}

Let $\S_n$ denote the symmetric group of order $n$, that is, the group
of permutations on $\{1, \dots, n\}$ under composition.  It is a
well-known fact that every permutation $\sigma \in \S_n$ can be
uniquely decomposed as a product of cycles. The \emph{cycle type} of
$\sigma$ is the sequence of natural numbers $\sigma_1, \sigma_2,
\sigma_3, \dots$ where $\sigma_i$ is the number of $i$-cycles in the
cycle decomposition of $\sigma$.  For example, the permutation
$(132)(45)(78)(6)$ has cycle type $1,2,1,0,0,0,\dots$

The \emph{exponential generating function} (egf) associated to a
combinatorial species $F$ is defined by \[ F(x) = \sum_{n \geq 0}
|F[n]| \frac{x^n}{n!}. \] That is, the coefficient of $x^n/n!$ is the
number of distinct $F$-structures on $n$ labels. For example, for $n
\geq 1$ there are $(n-1)!$ distinct labelled cycles (necklaces), so
the egf for the species of cycles is \[ C(x) = \sum_{n \geq 1} (n-1)!
\frac{x^n}{n!} = \sum_{n \geq 1} \frac{x^n}{n}. \]

The \emph{ordinary generating function} (ogf) associated to a species
$F$ is defined by \[ \tilde F(x) = \sum_{n \geq 0} |F[n]/\mathord{\sim}| x^n \]
where $\sim$ is the equivalence relation on $F$-structures induced by
permuting the labels.  That is, $\tilde F$ counts \emph{equivalence
  classes} of $F$-structures up to relabelling.  For example, there is
only one equivalence class of cycles for each $n$ (any two cycles
``have the same shape'' and are thus related by some relabelling), so
the ogf for the species of cycles is \[ \tilde C(x) = \sum_{n \geq 1}
x^n = \frac{x}{1-x}. \]

Egfs are quite natural, and the mapping from species to their
associated egf is a homomorphism that preserves many operations such
as sum, product, composition, and derivative.  ogfs, it turns out, are
not quite as nice; the mapping from species to ogfs preserves sum and
product but does not, in general, preserve composition or derivative.
In some sense ogfs throw away too much information.  Thus it is not
possible to compositionally determine the ogf for a species defined in
terms of these operations---at least not directly.  The solution is to
turn to a richer class of generating functions that do preserve the
necessary information.

\section{Cycle index series}
\label{sec:cycle-inde}

For a species $F$ and a permutation $\sigma \in \S_n$, let $\Fix
F[\sigma]$ denote the number of $F$-structures that are fixed by the
action of $\sigma$, that is, \[ \Fix F[\sigma] = \#\{ f \in F[n] \mid
F[\sigma] f = f \}. \] The \emph{cycle index series} of a
combinatorial species $F$ is a power series in an infinite set of
variables $x_1, x_2, x_3, \dots$ defined by \[ Z_F(x_1, x_2, x_3,
\dots) = \sum_{n \geq 0} \frac{1}{n!}  \sum_{\sigma \in \S_n} \Fix
F[\sigma] x_1^{\sigma_1} x_2^{\sigma_2} \dots \] We also sometimes
write $x^\sigma$ as an abbreviation for $x_1^{\sigma_1} x_2^{\sigma_2}
x_3^{\sigma_3} \dots$. As a simple example, consider the species of
lists, \ie linear orderings.  For each $n$, the identity permutation
(with cycle type $n,0,0,\dots$) fixes all $n!$ lists of length $n$,
whereas all other permutations do not fix any lists.  Therefore \[
Z_L(x_1, x_2, x_3, \dots) = \sum_{n \geq 0} \frac{1}{n!}  n! x_1^n =
\sum_{n \geq 0} x_1^n = \frac{1}{1 - x_1}. \]

Cycle index series are linked to both egfs and ogfs by the identities
\begin{gather}
  F(x) = Z_F(x,0,0,\dots) \label{eq:ci-egf} \\
  \tilde F(x) = Z_F(x,x^2,x^3, \dots) \label{eq:ci-ogf}
\end{gather}
To show \eqref{eq:ci-egf}, note that setting all $x_i$ to $0$ other
than $x_1$ means that the only terms that survive are terms with only
$x_1$ raised to some power.  These correspond to permutations with
only $1$-cycles, that is, identity permutations.  Identity
permutations fix \emph{all} $F$-structures of a given size, so we have
\begin{align*}
  Z_F(x,0,0,\dots) &= \sum_{n \geq 0} \frac{1}{n!} \Fix F[\mathit{id}]
  x^n \\
  &= \sum_{n \geq 0} |F[n]| \frac{x^n}{n!}.
\end{align*}

The proof of \eqref{eq:ci-ogf} depends on Burnside's Lemma.  Note that
for any permutation $\sigma \in \S_n$ with cycle type
$\sigma_1,\sigma_2,\sigma_3,\dots$ we have $\sigma_1 + 2\sigma_2 +
3\sigma_3 + \dots = n$.  Thus:
\begin{align*}
  Z_F(x,x^2,x^3,\dots) &= \sum_{n \geq 0} \frac{1}{n!} \sum_{\sigma
    \in \S_n} \Fix F[\sigma]
  x^{\sigma_1}x^{2\sigma_2}x^{3\sigma_3}\dots \\
  &= \sum_{n \geq 0} \frac{1}{n!} \sum_{\sigma \in \S_n} \Fix
  F[\sigma] x^n \\
  &= \sum_{n \geq 0} |F[n]/\mathord{\sim}| x^n
\end{align*}
where the final step is an application of Burnside's Lemma.

Crucially, the mapping from species to cycle index series is again a
homomorphism for many of the operations we care about, including
composition.  So one can compute with cycle index series and project
down to ogfs at the end.

\section{Cycle index series for bracelets}
\label{sec:bracelet-cycle-index}

We now work out the cycle index series for the species $B$ of
bracelets, where a bracelet structure is construed as a sequence of
(uniquely labelled) beads, considered equivalent up to rotation and
reflection.

Note first that bracelets of size $n$ have the dihedral group $D_{2n}$
as their symmetry group.  That is, every $n$-bracelet is fixed by the
action of any element of $D_{2n}$, and no bracelets are fixed by the
action of any other permutation.  There is a single $n$-bracelet for
each $n \in \{1,2\}$ (by convention there are no $0$-bracelets, just
as there are no $0$-cycles/necklaces), and for $n \geq 3$ there are
$(n-1)!/2$ labelled bracelets.

Thus, the cycle index series is given by
\begin{align*}
Z_B(x_1, x_2, x_3, \dots)
&= x_1 + \frac{1}{2}(x_1^2 + x_2) + \sum_{n \geq 3} \frac{1}{n!}
\sum_{\sigma \in D_{2n}} \frac{(n-1)!}{2} x^\sigma \\
&= x_1 + \frac{1}{2}(x_1^2 + x_2) + \frac{1}{2} \sum_{n \geq 3}
\sum_{\sigma \in D_{2n}} x^\sigma.
\end{align*}
Our remaining task is thus to compute $\sum_{\sigma \in D_{2n}}
x^\sigma$, that is, to compute the cycle types of elements of
$D_{2n}$.



\end{document}
